\documentclass[12pt]{article}
\usepackage[russian]{babel}
\usepackage[utf8x]{inputenc}
\usepackage{amssymb}
\usepackage{amsmath}
\usepackage{graphicx}
\usepackage[margin=.7in]{geometry}
\usepackage[colorinlistoftodos]{todonotes}
\usepackage{listings}
\usepackage[section]{placeins}
\usepackage[T1]{fontenc}
\author{}

\begin{document}
\title{Задачи}
\date{}
\maketitle

\section{Интерполяционно-матричная задача}
$\mathbb{R}^n: \{a_1, ..., a_n\} \rightarrow \{b_1, ..., b_n\}$ \\
$\mathbb{R}^4$ СЛАУ \\\\
$
\bigg\{ 
  \begin{tabular}{l}
    \( A a_1 = b_1 \) \\
    \( A a_n = b_n \) \\
  \end{tabular}
$
Разложить на базис

\section{Решение систем дифференциальных уравнений 2-ого порядка методом Лапласа}
\section{Три точки в три точки на комплексной плоскости. Дробно-линейное преобразование}
Класс конформных отображений. \\
Дробно-линейное преобразование
$f(z) = \frac{az + b}{cz + d}$
\\
Частный случай: Преобразование Мёбиуса:
$f(z) = \frac{az + b}{cz + d}$\\
Проекция сферы Римана на комплексную плоскость

$
  \frac{(z - z_1)(z_3 - z_2)}{(z - z_2)(z_3 - z_1)} = 
  \frac{(w - w_1)(w_3 - w_2)}{(w - w_2)(w_3 - w_1)}
$
\\\\
\textbf{Правило симметричности}
Точки $z_1$ и $z_2$ симметричны относительно окружности $|z - z_0| = R$ тогда, когда:\\
$|z_1 - z_0||z_2 - z_0| = R^2$


\end{document}