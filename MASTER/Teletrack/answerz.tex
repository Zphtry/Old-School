\documentclass[12pt]{article}
\usepackage[russian]{babel}
\usepackage[utf8x]{inputenc}
\usepackage{amssymb}
\usepackage{amsmath}
\usepackage{graphicx}
\usepackage[margin=.7in]{geometry}
\usepackage[colorinlistoftodos]{todonotes}
\usepackage{listings}
\usepackage[section]{placeins}
\usepackage[T1]{fontenc}



\begin{document}
\title{Вопросы}
\date{}
\maketitle

\begin{enumerate}

	\item \textbf{Однозвенный коммутатор. Условие блокировки для о.к. Структура и способы реализации о.к.} \\ 
	Имеются $M$ входов и $N$ выходов.\\
	Каждому входу соответствует интенсивность поступления нагрузки $\lambda$, выходу интенсивность её ухода $\mu$. 
	О.к. блокируется и поток сделавший запрос получает отказ, когда не осталось свободных выходов. \\
	Нет перегрузок и внутренних блокировок

	\item \textbf{Распределение Энгсета} \\
	Для $c$ установленных соединений:
	 \[
		 p_c = \frac{\rho ^ c \binom{M}{c}}
		 {\sum_{n=0}^{N} \rho ^ n \binom{M}{n}}
	 \]

	\item \textbf{Вероятность потерь по времени и по вызовам} \\
	По времени (предельный Энгсет):
	\[
 		p = \frac{\rho ^ N \binom{M}{N}}
 		{\sum_{n=0}^{N} \rho ^ n \binom{M}{n}}
	\]

	По вызовам (смещённый на единицу назад предельный Энгсет):
	\[
 		p = \frac{\rho ^ N \binom{M - 1}{N}}
 		{\sum_{n=0}^{N} \rho ^ n \binom{M - 1}{n}}
	\]

		
	\item \textbf{Распределение Эрланга} \\
	Для $c$ установленных соединений:
	\[
	 	p_c = \frac{A ^ c}
	 	{N!\sum_{n=0}^{N} \frac{A ^ n}{n!}}, A = \frac{\lambda}{\mu}
	\]

	\item \textbf{Термин «эрланг» и варианты трактовки } \\
	«Эрланг» - единица измерения нагрузки на линию. Вычисляется как соотношение поступающей нагрузки к скорости обработки коммутатором запроса (?).

	\item \textbf{Биноминальное распределение и физ. смысл}
		\[
		 	p = \binom{M}{N} \alpha ^ N (1 - \alpha) ^ {M - N}
		\]
	\item \textbf{Особенности моделирования коммутаторов при различных соотношениях $M$ к $N$: $M = N, M > N, M ≫ N$} \\
	\begin{itemize}
		\item При $M = N$ биноминальное
		\item При $M > N$ Энгсет
		\item При $M ≫ N$ Эрланг
	\end{itemize}
	
	\item \textbf{Явные потери и модели систем с явными потерями} \\
	Хз, честно говоря
	\item \textbf{Вторая формула Эрланга} \\
	Предельный случай, когда все линии заняты.
	\[
	 	p = \frac{A ^ N}
	 	{N!\sum_{n=0}^{N} \frac{A ^ n}{n!}}, A = \frac{\lambda}{\mu}
	\]

	\item \textbf{Производительность и среднее число соединений.
	Соотношение между производительностью и нагрузкой.} \\
		Производительность $G = \lambda M(1-p)$ \\
		Среднее число соединений $E = \rho M(1-p)$

\end{enumerate}

\end{document}